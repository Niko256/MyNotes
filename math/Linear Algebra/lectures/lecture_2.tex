\section{Ортонормированные базисы в $E$ и $U$}

$\mathbb{V}$ - это $E$ или $V$

\begin{shdef}
    \begin{definition}
        $x \perp y$, если $\langle x, y \rangle = 0$
    \end{definition}
    
    \begin{definition}
        $\mathbb{E}$ - базис в $\mathbb{V}$. \quad
        $\mathbb{E}$ - ОНБ, \ если $\langle e_{i}, e_{j} \rangle = \delta_{ij} = \begin{cases}
        1, & i = j \\
        0, & i \neq j \\
        \end{cases}$
    \end{definition}
\end{shdef}

\vspace{0.4cm}
\begin{shth}
    \begin{theorem}
        Если $a_{1}, \ldots, a_{k} \neq 0$ и $a_{i} \perp a_{j}, \ i \neq j,$ то это ЛНЗ система.
    \end{theorem}
\end{shth}

\begin{proof}
\leavevmode \newline

    От противного: Они ЛЗ $\Longrightarrow$ $\exists \ \alpha_{1}, \ldots, \alpha_{k}$ не все 0, что: 
\newline 

    $$(*) \ \alpha_{1} a_{1} + \ldots + \alpha_{k} a_{k} = 0$$
    Умножаем на $a_{1}$
    \newline
    
    $\Longrightarrow \alpha_{1} \langle a_{1}, a_{1} \rangle + \alpha_{2} \langle a_{2}, a_{1} \rangle + \ldots + \alpha_{k} \langle a_{k}, a_{1} \rangle = 0 \ \Longrightarrow \alpha_{1} = 0$
    \newline
    
    умножаем $(*)$ на $a_{2}$ и получаем, что $\alpha_{2} = 0$
    
    \
    
    $\Longrightarrow$ все $\alpha_{i} = 0 \ \Longrightarrow$ противоречие! 
\end{proof}
\vspace{0.6cm}

\begin{shth}
    \begin{theorem}
        (Ортогонализация по Шмидту)
        
        Пусть $\mathbb{V}$ - это $E$ или $V$, тогда:
        \begin{enumerate}
            \item ОНБ существует
            \item если ${(f_{1}, \ldots f_{n})}$ - базис в $\mathbb{V}$, то ОНБ можно построить по формуле:
            
            \[
            \begin{cases}
                g_{1} = f_{1}, & e_{1} = \frac{g_{1}}{\left\|g_{1}\right\|} \\
                g_{2} = f_{2} - \left(f_{2}, e_{1}\right) \cdot e_{1}, & e_{2} = \frac{g{2}}{\left\|g_{2}\right\|} \\
                \vdots \\
                g_{n} = f_{n} - \left(f_{n}, e_{1}\right) \cdot e_{1} - \left(f_{n}, e_{2}\right) \cdot e_{2} \ldots - \left(f_{n}, e_{n-1}\right) \cdot e_{n-1}, & e_{n} = \frac{g_{n}}{\left\|g_{n}\right\|}
            \end{cases}
            \]
        \end{enumerate}
    \end{theorem}
\end{shth}

\clearpage

\begin{shth}
    \begin{lemma}
        \[ \forall k \in {\{1, \ldots, n \}} \quad g_{k} \in \mathcal{L}(f_{1}, \ldots f_{k}), \qquad g_{k} \neq 0 \]
    \end{lemma}
\end{shth}

\vspace{0.4cm}
\begin{proof}
    по индукции:
    
    \begin{enumerate}
    
    
        \item $ k = 1 \quad g_{1} = f_{1} \in \mathcal{L}(f_{1}), \quad f_{1} \neq 0$
        
        
        \item $k = m - 1  - \text{верно}, \; k = m ?$ 
        \nl

        $g_{m} = f_{m} - \inner{f_{m}}{e_{1}} \; e_{1} - \ldots - \inner{f_{m}}{e_{m-1}} \; e_{m-1} = f_{m} - \alpha_{1} g_{1} - \ldots - \alpha_{m-1} g_{m-1}$
        \nl
        
        Так как все $g_{i}$ линейно выражаются через $f_{1}, \ldots, f_{m-1}$
        \nl
        
        $\lra  f_{m} - \alpha_{1} g_{1} - \ldots - \alpha_{m-1} g_{m-1} = f_{m} - \beta_{1}f_{1} - \ldots \beta_{m-1}f_{m-1} \quad  \nl \lra g_{m} \in \mathcal{L}(f_{1}, \ldots, f_{m})$        
        \nl 
        
        если $g_{m} = 0 \lra f_{m} - \beta_{1}f_{1} - \ldots - \beta_{m-1}f_{m-1} = 0$
        \nl 
        
        Это нетривиальная линейная комбинация равная нулю $\lra$ противоречие, так как $f_{1}, \ldots, f_{m}$ - базис линейной оболочки $\mathcal{L}(f_{1}, \ldots, f_{m})$ 
    \end{enumerate}
\end{proof}

\clearpage

Теперь вернёмся к доказательству Теоремы 5.
\begin{proof}

Осталось доказать, что все оставшиеся векторы попарно ортогональны.

Докажем по индукции, что для всех \( k \in \{1, \ldots, n\} \) векторы \( e_1, e_2, \ldots, e_k \) образуют ортогональную систему.

\begin{enumerate}
    \item База индукции: \( k = 1 \).

    По построению \( e_1 = \frac{g_1}{\|g_1\|} \), где \(g_1 = f_1\). Так как \(\|e_1\| = 1\), то \( e_1 \) — единичный вектор.

    \item Предположим, что для некоторого \( k = m-1 \) векторы \( e_1, e_2, \ldots, e_{m-1} \) образуют ортогональную систему.

    \item Шаг индукции: \( k = m \).

    Покажем, что \( e_m \) ортогонален каждому из \( e_1, e_2, \ldots, e_{m-1} \).

    Рассмотрим вектор \( g_m \):
    \[
    g_m = f_m - \sum_{i=1}^{m-1} \langle f_m, e_i \rangle e_i
    \]
    По построению \( e_m = \frac{g_m}{\|g_m\|} \).

    Для любого \( j \in \{1, 2, \ldots, m-1\} \) вычислим скалярное произведение \( \langle e_m, e_j \rangle \):
    \[
    \langle e_m, e_j \rangle = \left\langle \frac{g_m}{\|g_m\|}, e_j \right\rangle = \frac{1}{\|g_m\|} \langle g_m, e_j \rangle
    \]
    Вычислим \( \langle g_m, e_j \rangle \):
    \[
    \langle g_m, e_j \rangle = \left\langle f_m - \sum_{i=1}^{m-1} \langle f_m, e_i \rangle e_i, e_j \right\rangle
    \]
    Раскроем скалярное произведение:
    \[
    \langle g_m, e_j \rangle = \langle f_m, e_j \rangle - \sum_{i=1}^{m-1} \langle f_m, e_i \rangle \langle e_i, e_j \rangle
    \]
    По предположению индукции \( e_1, e_2, \ldots, e_{m-1} \) ортогональны, поэтому \newline \( \langle e_i, e_j \rangle = 0 \) для \( i \neq j \).
    \nl 
    
    Следовательно,  $
    \langle g_m, e_j \rangle = \langle f_m, e_j \rangle - \langle f_m, e_j \rangle = 0 $
    \nl 
    
    
    Таким образом, $\langle e_m, e_j \rangle = \frac{1}{\|g_m\|} \cdot 0 = 0$. 
    Это означает, что \( e_m \) ортогонален каждому из \( e_1, e_2, \ldots, e_{m-1} \).
    \nl 

    Следовательно, векторы \( e_1, e_2, \ldots, e_m \) образуют ортогональную систему.
\end{enumerate}

\vspace{0.4cm}

Таким образом, по индукции мы доказали, что векторы \( e_1, e_2, \ldots, e_n \) образуют ортогональную систему. А так как все векторы \( e_i \) нормированы (\(\|e_i\| = 1\)), то система \( e_1, e_2, \ldots, e_n \) является ортонормированным базисом.

\end{proof}

\begin{shdef}
    \begin{definition}
    \leavevmode \\

        Ортогональная матрица — это квадратная матрица $A$ с вещественными      элементами, удовлетворяющая условию: 
        $$A^T A = A A^T = E$$
        \nl 

        Унитарная матрица — это квадратная матрица $U$ с комплексными элементами, удовлетворяющая условию: 
        $$U^* U = U U^* = E$$
        \nl

        ($ U^* $ - транспонированная и комплексно-сопряженная матрица.)
    \end{definition}
\end{shdef}

\begin{shth}
\begin{theorem}
    \begin{enumerate}
    \leavevmode \nl 
    
        \item Пусть $\mathbb{V} = E$ — Евклидово пространство, $\mathcal{E}_1, \mathcal{E}_2$ — ортонормированные базисы. 
              $$T \; \text{ - матрица перехода, тогда} \; T \text{ - ортогональная}.$$
        \item Пусть $\mathbb{V} = U$ — Унитарное пространство, $\mathcal{E}_1, \mathcal{E}_2$ — ортонормированные базисы. 
              $$T \; \text{ - матрица перехода, тогда} \; T \text{ - унитарная}.$$
    \end{enumerate}
\end{theorem}
\end{shth}

\begin{proof}
    \leavevmode \nl  
    
    \begin{enumerate}
        \item  $ T = (t_{ij}), \quad \mathcal{E'} = \mathcal{E}T , \quad \mathcal{E} = \{e_{1}, \ldots, e_{n} \}, \quad \mathcal{E'} = \{e'_{1}, \ldots, e'_{n} \}$
        \nl 
        
        \quad Для ортонормированного базиса \( \mathcal{E}' \) должно выполняться условие \( \delta_{ij} = \langle e'_i, e'_j \rangle \).
        \nl
        
        \quad Выразим \( e'_i \) и \( e'_j \) через старый базис \( \mathcal{E} \):
        \[
        e'_i = \sum_{k=1}^n t_{ki} e_k, \quad e'_j = \sum_{m=1}^n t_{mj} e_m.
        \]
        \nl
        
        \quad Тогда скалярное произведение \( \langle e'_i, e'_j \rangle \) можно записать как:
        \[
        \delta_{ij} = \langle e'_i, e'_j \rangle = \left\langle \sum_{k=1}^n t_{ki} e_k, \sum_{m=1}^n t_{mj} e_m \right\rangle.
        \]
        \nl
        
        \quad Используя свойства скалярного произведения, получим:
        \[
        \delta_{ij} = \sum_{k=1}^n \sum_{m=1}^n t_{ki} t_{mj} \langle e_k, e_m \rangle.
        \]
        \nl
        
        \quad Поскольку \( \mathcal{E} \) — ортонормированный базис, \( \langle e_k, e_m \rangle = \delta_{km} \), и следовательно:
        \[
        \delta_{ij} = \sum_{k=1}^n t_{ki} t_{kj}.
        \]
        \nl
        
        \quad Это означает, что \( T^T T = E \), где \( E \) — единичная матрица. Таким образом, матрица \( T \) ортогональна.
        \nl 
    
        \item $ T = (t_{ij}), \quad \mathcal{E'} = \E T, \quad \E = {e_{1}, \ldots, e_{n}}, \; \mathcal{E'} = \{e'_{1}, \ldots, e'_{n}\} $
        \nl
        
        \quad Для ортонормированного базиса \( \mathcal{E}' \) должно выполняться условие \( \delta_{ij} = \langle e'_i, e'_j \rangle \).
        \nl
        
        \quad Выразим \( e'_i \) и \( e'_j \) через старый базис \( \mathcal{E} \):
        \[
        e'_i = \sum_{k=1}^n t_{ki} e_k, \quad e'_j = \sum_{m=1}^n t_{mj} e_m.
        \]
        \nl
        
        \quad Тогда скалярное произведение \( \langle e'_i, e'_j \rangle \) можно записать как:
        \[
        \delta_{ij} = \langle e'_i, e'_j \rangle = \left\langle \sum_{k=1}^n t_{ki} e_k, \sum_{m=1}^n t_{mj} e_m \right\rangle.
        \]
        \nl
        
        \quad Используя свойства скалярного произведения, получим:
        \[
        \delta_{ij} = \sum_{k=1}^n \sum_{m=1}^n t_{ki} \overline{t_{mj}} \langle e_k, e_m \rangle.
        \]
        \nl
        
        \quad Поскольку \( \mathcal{E} \) — ортонормированный базис, \( \langle e_k, e_m \rangle = \delta_{km} \), и следовательно:
        \[
        \delta_{ij} = \sum_{k=1}^n t_{ki} \overline{t_{kj}}.
        \]
        \nl
        
        \quad Это означает, что \( T^* T = E \), где \( E \) — единичная матрица. Таким образом, матрица \( T \) унитарная.
    \end{enumerate}
\end{proof}


\section{Ортогональное дополнение подпространств}

$\mathbb{V}$ - это $E$ или $U, \quad V_{1}, V_{2} \subseteq \mathbb{V}$ - подпространства.
        
\begin{shdef}
    \begin{definition}
    \leavevmode \\

        Ортогональное дополнение подпространства. 
        
        $V_{1} \bot V_{2}$, если $\forall x \in V_{1}, \forall y \in V_{2} \; \inner{x}{y} = 0$
    \end{definition}
\end{shdef}

\begin{shth}
    \begin{theorem}
        $V_{1} \bot V_{2} \lra V_{1} \cap V_{2} = \{0\}$
    \end{theorem}
\end{shth}

\begin{proof}
    \leavevmode \nl
    
    пусть $z \in V_{1} \cap V_{2} \; z \in V_{1}, z \in V_{2}, \; \inner{z}{z} = 0 \quad \lra z = 0$
\end{proof}

\begin{shdef}
    \begin{definition}
        $\mathbb{V}_{1} \subseteq \mathbb{V} \Longrightarrow \mathbb{V}_{1}^\bot = \{ z \in \mathbb{V}: \inner{y}{z} = 0 \quad \forall y \in \mathbb{V}_{1} \}$
    \end{definition}
\end{shdef}

\begin{shth}
    \begin{theorem}
        $\mathbb{V}_{1}^\bot$ — подпространство.
    \end{theorem}
\end{shth}

\begin{proof}
    \leavevmode \nl
    
    \begin{enumerate}
        \item $\vec{0} \in \mathbb{V}_{1}^\bot \Leftrightarrow \mathbb{V}_{1}^\bot \neq \vec{0}$
        \item $z_{1}, z_{2} \in \mathbb{V}_{1}^\bot \quad x = \alpha z_{1} + \beta z_{2}, \quad \forall y \in \mathbb{V}$ \\
        
        $\langle y, x \rangle = \langle y, \alpha z_{1} + \beta z_{2} \rangle = \overline{\alpha} \langle y, z_{1} \rangle + \overline{\beta} \langle y, z_{2} \rangle = 0. \quad \lra x \in \mathbb{V}_{1}^\bot$
    \end{enumerate}
\end{proof}


\begin{shth}
    \begin{theorem}
        \[\mathbb{V}_{1}, \mathbb{V}_{1}^\bot : \mathbb{E}_{1} = \{e_{1}, \ldots, e_{n}\} \text{ — ОНБ в } \mathbb{V}_{1}. \quad \text{Тогда } z \in \mathbb{V}_{1}^{\bot} \lra \inner{z}{e_{1}} = \ldots = \inner{z}{e_{k}} = 0\]
    \end{theorem}
\end{shth}

\begin{proof}
    \leavevmode \nl
    
    $ \boxed{\lra} \quad \text{Следует из определения} $
    
    $ \boxed{\lla} \quad \inner{z}{e_{1}} = \ldots \inner{z}{e_{k}} = 0$
    
    $ \forall y \in \mathbb{V}_{1}: \; y = \alpha_{1}e_{1} + \ldots + \alpha_{n}e_{n}$
    
    $ \inner{z}{y} = \inner{z}{\alpha_{1}e_{1} + \ldots + \alpha_{n}e_{n}} = \overline{\alpha_{1}} \inner{z}{e_{1}} + \ldots + \overline{\alpha_{n}} \inner{z}{e_{n}} = 0$
    
\end{proof}

\clearpage