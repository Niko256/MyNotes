\chapter{Функции от матриц}

\section{Вычисление многочленов от матриц с помощью \\Жордановой формы.}

Пусть \( A \) — матрица оператора \( \phi \), все корни характеристического уравнения вещественные. Рассмотрим степень матрицы \( A \):

\[
A^s = \underbrace{A \cdot A \cdot \ldots \cdot A}_{s \text{ раз}}.
\]

\begin{shth}
    \begin{theorem}
        Пусть \( T \) — матрица перехода от \( A \) к её Жордановой форме \( J \), то есть
        \[
        A = T J T^{-1}.
        \]
        Тогда для любого натурального числа \( s \) выполняется:
        \[
        A^s = T J^s T^{-1}.
        \]
    \end{theorem}
\end{shth}

\begin{proof}
    Докажем теорему по индукции.

    \textbf{База индукции:} Для \( s = 1 \) утверждение очевидно, так как
    \[
    A^1 = A = T J T^{-1}.
    \]

    \textbf{Предположение индукции:} Пусть для некоторого \( s = k \) утверждение верно, то есть
    \[
    A^k = T J^k T^{-1}.
    \]

    \textbf{Шаг индукции:} Докажем, что утверждение верно для \( s = k + 1 \). Рассмотрим:
    \[
    A^{k+1} = A^k \cdot A.
    \]
    Подставим выражение для \( A^k \) из предположения индукции:
    \[
    A^{k+1} = (T J^k T^{-1}) \cdot A.
    \]
    Заметим, что \( A = T J T^{-1} \), поэтому:
    \[
    A^{k+1} = (T J^k T^{-1}) \cdot (T J T^{-1}).
    \]
    Упростим выражение, учитывая, что \( T^{-1} T = E \):
    \[
    A^{k+1} = T J^k (T^{-1} T) J T^{-1} = T J^k E J T^{-1} = T J^k J T^{-1}.
    \]
    Поскольку \( J^k J = J^{k+1} \), получаем:
    \[
    A^{k+1} = T J^{k+1} T^{-1}.
    \]

    Таким образом, по индукции утверждение верно для любого натурального \( s \).
\end{proof}

\begin{shdef}
    \begin{definition}
        \textbf{Степень Жордановой формы.} 
        
        Пусть \( J \) — Жорданова форма матрицы \( A \), состоящая из Жордановых клеток \( J_1, J_2, \ldots, J_m \). Тогда степень \( J^s \) определяется как блочно-диагональная матрица:
        \[
        J^s = 
        \begin{pmatrix}
        J_1^s & 0 & \cdots & 0 \\
        0 & J_2^s & \cdots & 0 \\
        \vdots & \vdots & \ddots & \vdots \\
        0 & 0 & \cdots & J_m^s
        \end{pmatrix},
        \]
        где каждая \( J_i^s \) — степень соответствующей Жордановой клетки.
    \end{definition}
\end{shdef}


\begin{shdef}
    \begin{definition}
        \textbf{Степень Жордановой клетки.} 
        
        Пусть \( J_k(\lambda) \) — Жорданова клетка размера \( k \times k \), соответствующая собственному значению \( \lambda \). Тогда её степень \( J_k(\lambda)^s \) определяется как:
        \[
        J_k(\lambda)^s = 
        \begin{pmatrix}
        \lambda^s & s \lambda^{s-1} & \frac{s(s-1)}{2!} \lambda^{s-2} & \cdots & \frac{s(s-1)\cdots(s-(k-2))}{(k-1)!} \lambda^{s-k+1} \\
        0 & \lambda^s & s \lambda^{s-1} & \cdots & \frac{s(s-1)\cdots(s-(k-3))}{(k-2)!} \lambda^{s-k+2} \\
        \vdots & \vdots & \ddots & \ddots & \vdots \\
        0 & 0 & \cdots & \lambda^s & s \lambda^{s-1} \\
        0 & 0 & \cdots & 0 & \lambda^s
        \end{pmatrix},
        \]
        где:
        \begin{itemize}
            \item На диагонали стоят элементы \( \lambda^s \).
            \item На \( r \)-й наддиагонали (\( r = 1, 2, \ldots, k-1 \)) стоят элементы вида:
            \[
            \frac{s(s-1)\cdots(s-r+1)}{r!} \lambda^{s-r}.
            \]
            \item Коэффициент \( \frac{s(s-1)\cdots(s-r+1)}{r!} \) — это биномиальный коэффициент \( C_s^r \).
        \end{itemize}
    \end{definition}
\end{shdef}

\clearpage

\begin{shdef}
    \begin{definition}
        \textbf{Многочлен от матрицы.} 
        
        Пусть \( P(t) \) — многочлен, а \( A \) — квадратная матрица, которая может быть приведена к Жордановой форме \( J \) с помощью матрицы перехода \( T \).
        
        Тогда \textbf{многочлен от матрицы} \( P(A) \) определяется как:
        \[
        P(A) = T \cdot P(J) \cdot T^{-1},
        \]
        где \( P(J) \) — блочно-диагональная матрица, состоящая из многочленов от Жордановых клеток \( J_1, J_2, \ldots, J_k \):
        \[
        P(J) = 
        \begin{pmatrix}
        P(J_1) & 0 & \cdots & 0 \\
        0 & P(J_2) & \cdots & 0 \\
        \vdots & \vdots & \ddots & \vdots \\
        0 & 0 & \cdots & P(J_k)
        \end{pmatrix}.
        \]
        Здесь \( P(J_i) \) — это многочлен от \( i \)-й Жордановой клетки \( J_i \), который вычисляется по формуле:
        \[
        P(J_i) = 
        \begin{pmatrix}
        P(\lambda_i) & \frac{P'(\lambda_i)}{1!} & \frac{P''(\lambda_i)}{2!} & \cdots & \frac{P^{(m_i-1)}(\lambda_i)}{(m_i-1)!} \\
        0 & P(\lambda_i) & \frac{P'(\lambda_i)}{1!} & \cdots & \frac{P^{(m_i-2)}(\lambda_i)}{(m_i-2)!} \\
        \vdots & \vdots & \ddots & \ddots & \vdots \\
        0 & 0 & \cdots & P(\lambda_i) & \frac{P'(\lambda_i)}{1!} \\
        0 & 0 & \cdots & 0 & P(\lambda_i)
        \end{pmatrix},
        \]
        где:
        \begin{itemize}
            \item \( \lambda_i \) — собственное значение, соответствующее \( i \)-й Жордановой клетке \( J_i \),
            \item \( m_i \) — размер \( i \)-й Жордановой клетки \( J_i \),
            \item \( P^{(k)}(\lambda_i) \) — \( k \)-я производная многочлена \( P(t) \), вычисленная в точке \( \lambda_i \).
        \end{itemize}
    \end{definition}
\end{shdef}

\begin{shdef}
    \begin{definition}
        \textbf{Равенство многочленов на спектре матрицы.} 
        
        Пусть \( P(t) \) и \( Q(t) \) — многочлены, \( A \) — квадратная матрица с собственными значениями \( \lambda_1, \lambda_2, \ldots, \lambda_s \). Пусть \( h_j \) — максимальная размерность Жордановой клетки, соответствующей собственному значению \( \lambda_j \). 

        Говорят, что многочлены \( P(t) \) и \( Q(t) \) \textbf{равны на спектре матрицы \( A \)}, если для каждого \( j = 1, 2, \ldots, s \) выполняются следующие условия:
        \begin{enumerate}
            \item Значения многочленов совпадают в точке \( \lambda_j \):
            \[
            Q(\lambda_j) = P(\lambda_j).
            \]
            \item Значения производных многочленов до порядка \( h_j - 1 \) включительно \\совпадают в точке \( \lambda_j \):
            \[
            Q^{(k)}(\lambda_j) = P^{(k)}(\lambda_j), \quad \forall k \in [1, 2, \ldots, h_j - 1],
            \]
            где \( Q^{(k)} \) и \( P^{(k)} \) — \( k \)-е производные многочленов \( Q(t) \) и \( P(t) \) соответственно.
        \end{enumerate}
    \end{definition}
\end{shdef}

\section{Многочлен Лагранжа-Сильвестра.}

\begin{shth}
    \begin{theorem}
        \textbf{О существовании и единственности интерполяционного многочлена Лагранжа-Сильвестра.} 
        
        Пусть \( f(t) \) — функция, определённая на спектре матрицы \( A \). Тогда существует единственный многочлен \( r_f(t) \), такой что:
        \begin{enumerate}
            \item Степень многочлена \( r_f(t) \) строго меньше степени минимального многочлена матрицы \( A \):
            \[
            \deg r_f(t) < \deg m_A(t),
            \]
            где \( m_A(t) \) — минимальный многочлен матрицы \( A \).
            \item Многочлен \( r_f(t) \) совпадает с функцией \( f(t) \) на спектре матрицы \( A \), то есть:
            \[
            r_f(\lambda_j) = f(\lambda_j), \quad \text{для всех } j = 1, 2, \ldots, s,
            \]
            и, если \( h_j \) — максимальная размерность Жордановой клетки, соответствующей собственному значению \( \lambda_j \), то:
            \[
            r_f^{(k)}(\lambda_j) = f^{(k)}(\lambda_j), \quad \text{для всех } k = 1, 2, \ldots, h_j - 1.
            \]
        \end{enumerate}
        Тогда многочлен \( r_f(t) \) называется \textbf{интерполяционным многочленом Лагранжа-Сильвестра} для функции \( f(t) \) на спектре матрицы \( A \).
    \end{theorem}
\end{shth}


\begin{proof}
\leavevmode \nl 

Рассмотрим минимальный многочлен матрицы \( A \):
\[
m_A(t) = (t - \lambda_1)^{h_1} \cdot (t - \lambda_2)^{h_2} \cdot \ldots \cdot (t - \lambda_s)^{h_s},
\]
где \( \lambda_1, \lambda_2, \ldots, \lambda_s \) — различные собственные значения матрицы \( A \), а \( h_j \) — максимальная размерность Жордановой клетки, соответствующей собственному значению \( \lambda_j \). Степень минимального многочлена равна:
\[
\deg m_A(t) = h_1 + h_2 + \ldots + h_s = h.
\]

Построим интерполяционный многочлен \( r_f(t) \), который удовлетворяет следующим условиям интерполяции:
\begin{enumerate}
    \item \( \deg r_f(t) < h \),
    \item \( r_f(\lambda_j) = f(\lambda_j) \) для всех \( j = 1, 2, \ldots, s \),
    \item \( r_f^{(k)}(\lambda_j) = f^{(k)}(\lambda_j) \) для всех \( k = 1, 2, \ldots, h_j - 1 \).
\end{enumerate}

Пусть \( r_f(t) \) имеет вид:
\[
r_f(t) = c_0 + c_1 t + c_2 t^2 + \ldots + c_{h-1} t^{h-1}.
\]
Условия интерполяции приводят к системе линейных уравнений относительно коэффициентов \( c_0, c_1, \ldots, c_{h-1} \). Для каждого собственного значения \( \lambda_j \) и каждого \( k = 0, 1, \ldots, h_j - 1 \) получаем уравнение:
\[
r_f^{(k)}(\lambda_j) = f^{(k)}(\lambda_j).
\]

Рассмотрим \textbf{однородную} систему, соответствующую условиям интерполяции:
\[
r_f^{(k)}(\lambda_j) = 0 \quad \forall j \in [1, 2, \ldots, s] \text{ и } \forall k \in [0, 1, \ldots, h_j - 1].
\]
Пусть \( \hat{P}(t) \) — многочлен, который удовлетворяет этой однородной системе. Тогда:
\begin{itemize}
    \item На спектре матрицы \( A \) многочлен \( \hat{P}(t) \) равен нулю, то есть \( \hat{P}(\lambda_j) = 0 \) и \( \hat{P}^{(k)}(\lambda_j) = 0 \) для всех \( j \) и \( k \).
    \item Поскольку функция от матрицы \( A \) определяется значениями на её спектре, то \( \hat{P}(A) = 0 \). Таким образом, \( \hat{P}(t) \) аннулирует матрицу \( A \).
    \item Однако минимальный многочлен \( m_A(t) \) также аннулирует матрицу \( A \), причём он имеет минимальную степень среди всех таких многочленов.
    \item Так как \( \deg \hat{P}(t) < \deg m_A(t) \), то \( \hat{P}(t) \) не может быть ненулевым многочленом, аннулирующим \( A \). Следовательно, \( \hat{P}(t) = 0 \).
\end{itemize}

Таким образом, однородная система имеет только тривиальное решение \( \hat{P}(t) = 0 \). Это означает, что соответствующая неоднородная система (с правыми частями \( f^{(k)}(\lambda_j) \)) имеет единственное решение. Следовательно, коэффициенты \( c_0, c_1, \ldots, c_{h-1} \) определяются однозначно, и многочлен \( r_f(t) \) существует и единственен.

\textbf{Построение \( r_f(t) \):}
Поскольку система имеет единственное решение, мы можем однозначно построить многочлен \( r_f(t) \), удовлетворяющий всем условиям интерполяции. Этот многочлен называется \textbf{интерполяционным многочленом Лагранжа-Сильвестра} для функции \( f(t) \) на спектре матрицы \( A \).

\end{proof}

\begin{shth}
    \begin{theorem}[О виде многочлена Лагранжа-Сильвестра в случае простых корней]
    \leavevmode \nl 

    Пусть \( \lambda_{1}, \ldots, \lambda_{n} \) — собственные числа матрицы \( A \), все различные. 
    
    Характеристический многочлен матрицы \( A \) имеет вид:
    \[
    x_{A}(t) = (t - \lambda_{1}) \cdot (t - \lambda_{2}) \cdot \ldots \cdot (t - \lambda_{n}).
    \]
    Тогда интерполяционный многочлен Лагранжа-Сильвестра \( r_{f}(t) \) для функции \( f(t) \) на спектре матрицы \( A \) задаётся формулой:
    \[
    r_{f}(t) = \sum_{j = 1}^{n} f(\lambda_{j}) \cdot \frac{(t - \lambda_{1}) \cdot (t - \lambda_{2}) \cdot \ldots \cdot (t - \lambda_{n})}{(\lambda_{j} - \lambda_{1}) \cdot \ldots \cdot (\lambda_{j} - \lambda_{j-1}) (\lambda_{j} - \lambda_{j+1}) \cdot \ldots \cdot (\lambda_{j} - \lambda_{n})}.
    \]
    \end{theorem}
\end{shth}

\begin{proof}

    Рассмотрим минимальный многочлен матрицы \( A \):
    \[
    m_{A}(t) = x_{A}(t) = (t - \lambda_{1}) \cdot (t - \lambda_{2}) \cdot \ldots \cdot (t - \lambda_{n}).
    \]
    Степень \( m_{A}(t) \) равна \( n \). По условию теоремы, степень интерполяционного многочлена \( r_{f}(t) \) должна быть строго меньше \( n \), то есть \( \deg r_{f}(t) = n - 1 < n \).

    Покажем, что \( r_{f}(t) \) удовлетворяет условиям интерполяции. Для этого проверим, что \( r_{f}(\lambda_{i}) = f(\lambda_{i}) \) для всех \( i = 1, 2, \ldots, n \).

   
    Для любого \( \lambda_{i} \) (\( i = 1, \ldots, n \)) выполняется:
        \begin{align*}
        r_{f}(\lambda_{i}) &= \sum_{j = 1, j \neq i}^{n} f(\lambda_{j}) \cdot \frac{(\lambda_{i} - \lambda_{1}) \cdot \ldots \cdot (\lambda_{i} - \lambda_{i}) \cdot \ldots \cdot (\lambda_{i} - \lambda_{n})}{(\lambda_{j} - \lambda_{1}) \cdot \ldots \cdot (\lambda_{j} - \lambda_{j-1}) (\lambda_{j} - \lambda_{j+1}) \cdot \ldots \cdot (\lambda_{j} - \lambda_{n})} \\
        &\quad + f(\lambda_{i}) \cdot \frac{(\lambda_{i} - \lambda_{1}) \cdot \ldots \cdot (\lambda_{i} - \lambda_{n})}{(\lambda_{i} - \lambda_{1}) \cdot \ldots \cdot (\lambda_{i} - \lambda_{n})}.
    \end{align*}
    Заметим, что в первом слагаемом числитель обращается в ноль, так как \\\( (\lambda_{i} - \lambda_{i}) = 0 \), а второе слагаемое равно \( f(\lambda_{i}) \). Таким образом, \( r_{f}(\lambda_{i}) = f(\lambda_{i}) \).

    Теперь рассмотрим вспомогательные многочлены:
    \[
    \psi_{j}(t) = \frac{m_{A}(t)}{t - \lambda_{j}}.
    \]
    Тогда интерполяционный многочлен можно записать в виде:
    \[
    r_{f}(t) = \sum_{j=1}^{n} f(\lambda_{j}) \cdot \frac{\psi_{j}(t)}{\psi_{j}(\lambda_{j})}.
    \]
\end{proof}

\begin{shth}
    \begin{theorem}[О виде многочлена Лагранжа-Сильвестра в общем случае]
    \leavevmode \nl 
    
    Пусть \( m_A(t) \) — минимальный многочлен матрицы \( A \), который имеет вид:
    \[
    m_A(t) = (t - \lambda_1)^{h_1} \cdot (t - \lambda_2)^{h_2} \cdot \ldots \cdot (t - \lambda_s)^{h_s},
    \]
    где \( \lambda_1, \lambda_2, \ldots, \lambda_s \) — различные собственные значения матрицы \( A \).

    Определим вспомогательные многочлены \( \psi_j(t) \) для каждого \( j = 1, 2, \ldots, s \) следующим образом:
    \[
    \psi_j(t) = \frac{m_A(t)}{(t - \lambda_j)^{h_j}}.
    \]

    Тогда интерполяционный многочлен Лагранжа-Сильвестра \( r_f(t) \) для функции \( f(t) \) на спектре матрицы \( A \) задаётся формулой:
    \begin{multline*}
    r_f(t) = \sum_{j=1}^s \psi_j(t) \cdot \Bigg[ \left. \frac{f(t)}{\psi_j(t)} \right|_{t = \lambda_j} + \frac{1}{1!} \left. \left( \frac{f(t)}{\psi_j(t)} \right)' \right|_{t = \lambda_j} \cdot (t - \lambda_j) \; + \\
    + \ldots + \frac{1}{(h_j - 1)!} \left. \left( \frac{f(t)}{\psi_j(t)} \right)^{(h_j - 1)} \right|_{t = \lambda_j} \cdot (t - \lambda_j)^{h_j - 1} \Bigg].
    \end{multline*}
    \end{theorem}
\end{shth}

\begin{proof}
\leavevmode \nl 

    Докажем, что многочлен \( r_f(t) \) удовлетворяет условиям интерполяции на спектре матрицы \( A \).

       Степень \( r_f(t) \) меньше степени минимального многочлена \( m_A(t) \), то есть \\\( \deg r_f(t) < \deg m_A(t) \). Это означает, что дробь \( \frac{r_f(t)}{m_A(t)} \) является правильной.

       Поскольку \( \frac{r_f(t)}{m_A(t)} \) — правильная дробь, её можно разложить на элементарные дроби:
       \[
       \frac{r_f(t)}{m_A(t)} = \sum_{j=1}^{s} \left[ \frac{\alpha_{j1}}{(t - \lambda_j)^{h_j}} + \frac{\alpha_{j2}}{(t - \lambda_j)^{h_j - 1}} + \ldots + \frac{\alpha_{j h_j}}{t - \lambda_j} \right] \quad (*)
       \]

       Заметим, что \( m_A(t) = \psi_j(t) \cdot (t - \lambda_j)^{h_j} \). Подставим это в разложение (*):
       \[
       \frac{r_f(t)}{\psi_j(t)} = (t - \lambda_j)^{h_j} \cdot \left[ \frac{\alpha_{j1}}{(t - \lambda_j)^{h_j}} + \frac{\alpha_{j2}}{(t - \lambda_j)^{h_j - 1}} + \ldots + \frac{\alpha_{j h_j}}{t - \lambda_j} \right] + (t - \lambda_j)^{h_j} \cdot \sum_{\substack{i=1 \\ i \neq j}}^{s} \left[ \frac{\alpha_{i1}}{(t - \lambda_i)^{h_i}} + \ldots + \frac{\alpha_{i h_i}}{t - \lambda_i} \right]
       \]
      Упростим первое слагаемое:
        \begin{multline*}
        \frac{r_f(t)}{\psi_j(t)} = \left[ \alpha_{j1} + \alpha_{j2} (t - \lambda_j) + \alpha_{j3} (t - \lambda_j)^2 + \ldots + \alpha_{j h_j} (t - \lambda_j)^{h_j - 1} \right] \\
        + (t - \lambda_j)^{h_j} \cdot \sum_{\substack{i=1 \\ i \neq j}}^{s} \left[ \frac{\alpha_{i1}}{(t - \lambda_i)^{h_i}} + \frac{\alpha_{i2}}{(t - \lambda_i)^{h_i - 1}} + \ldots + \frac{\alpha_{i h_i}}{t - \lambda_i} \right] \quad (**)
        \end{multline*}
       Обозначим второе слагаемое как \( g_j(t) \).

       Заметим, что \( g_j(t) \) и её производные до порядка \( h_j - 1 \) обращаются в ноль при \( t = \lambda_j \):
       \[
       \begin{cases}
           g_j(\lambda_j) = 0, \\
           g_j'(\lambda_j) = 0, \\
           \vdots \\
           g_j^{(h_j - 1)}(\lambda_j) = 0 \quad (***)
       \end{cases}
       \]
       Это означает, что \( g_j(t) \) не влияет на интерполяцию в точке \( \lambda_j \).


       Подставим \( t = \lambda_j \) в выражение (**):
       \[
       \alpha_{j1} = \left. \frac{r_f(t)}{\psi_j(t)} \right|_{t = \lambda_j}.
       \]
       Продифференцируем (**) по \( t \) и подставим \( t = \lambda_j \):
       \[
       \alpha_{j2} = \frac{1}{1!} \left. \left( \frac{r_f(t)}{\psi_j(t)} \right)' \right|_{t = \lambda_j}
       \]
       Продолжая аналогично, получим:
       \[
       \alpha_{j h_j} = \frac{1}{(h_j - 1)!} \left. \left( \frac{r_f(t)}{\psi_j(t)} \right)^{(h_j - 1)} \right|_{t = \lambda_j}
       \]


       Поскольку \( r_f(t) \) совпадает с \( f(t) \) на спектре матрицы \( A \), коэффициенты \( \alpha_{jk} \) можно выразить через \( f(t) \):
       \[
       \alpha_{j1} = \left. \frac{f(t)}{\psi_j(t)} \right|_{t = \lambda_j}, \quad
       \alpha_{j2} = \frac{1}{1!} \left. \left( \frac{f(t)}{\psi_j(t)} \right)' \right|_{t = \lambda_j}, \quad \ldots, \quad
       \alpha_{j h_j} = \frac{1}{(h_j - 1)!} \left. \left( \frac{f(t)}{\psi_j(t)} \right)^{(h_j - 1)} \right|_{t = \lambda_j}
       \]


       Подставим найденные коэффициенты \( \alpha_{jk} \) в разложение (*):
       \[
       \frac{r_f(t)}{m_A(t)} = \sum_{j=1}^{s} \left[ \frac{\left. \frac{f(t)}{\psi_j(t)} \right|_{t = \lambda_j}}{(t - \lambda_j)^{h_j}} + \ldots + \frac{\frac{1}{(h_j - 1)!} \left. \left( \frac{f(t)}{\psi_j(t)} \right)^{(h_j - 1)} \right|_{t = \lambda_j}}{t - \lambda_j} \right]
       \]
       Умножив обе части на \( m_A(t) \), получим:
       \[
       r_f(t) = \sum_{j=1}^{s} \psi_j(t) \cdot \left[ \left. \frac{f(t)}{\psi_j(t)} \right|_{t = \lambda_j} + \ldots + \frac{1}{(h_j - 1)!} \left. \left( \frac{f(t)}{\psi_j(t)} \right)^{(h_j - 1)} \right|_{t = \lambda_j} \cdot (t - \lambda_j)^{h_j - 1} \right]
       \]

\end{proof}

\section{Представление функций от матриц с помощью степенных рядов}

\begin{shdef}
    \begin{definition}[Сходимость последовательности функций на спектре]
    \leavevmode \nl 
    
    Последовательность функций \( f_n(t) \) сходится к функции \( f(t) \) на спектре матрицы \( A \), если \(\forall i \in \{1, \ldots, s\} \) и \(\forall k \in \{0, 1, \ldots, h_i - 1\} \) выполняется:
    \[
    f_n^{(k)}(\lambda_i) \to f^{(k)}(\lambda_i) \quad \text{при } n \to \infty,
    \]
    где \( \lambda_i \) — собственные значения матрицы \( A \), а \( h_i \) — размерность наибольшей \\жордановой клетки, соответствующей \( \lambda_i \).
    \end{definition}
\end{shdef}


\begin{shdef}
    \begin{definition}[Функция от матрицы в предельной форме]
    \leavevmode \nl 

    Пусть \( f_n(t) \) — последовательность функций, определённых на спектре матрицы \( A \).
    
    Тогда функция от матрицы \( f(A) \) определяется как предел:
    \[
    f(A) = \lim_{n \to \infty} f_n(A),
    \]
    если последовательность \( f_n(t) \) сходится к функции \( f(t) \) на спектре матрицы \( A \).
    \end{definition}
\end{shdef}

\begin{shdef}
    \begin{definition}[Функция от матрицы через сходящийся ряд]
        \leavevmode \nl 

        Пусть \( \sum_{k=1}^{\infty} u_k(t) = S(t) \) — сходящийся ряд функций, определённых на спектре матрицы \( A \). Обозначим частичные суммы ряда как:
        \[
        \sigma_N(t) = \sum_{k=1}^{N} u_k(t).
        \]
        Если последовательность частичных сумм \( \sigma_N(t) \) сходится к \( S(t) \) на спектре \\матрицы \( A \), то есть для всех собственных значений \( \lambda_i \) матрицы \( A \) и \\\( \forall k = 0, 1, \ldots, h_i - 1 \) выполняется:
        \[
        \sigma_N^{(k)}(\lambda_i) \to S^{(k)}(\lambda_i) \quad \text{при } N \to \infty,
        \]
        то функция от матрицы \( S(A) \) определяется как сумма ряда:
        \[
        S(A) = \sum_{k=1}^{\infty} u_k(A).
        \]
    \end{definition}
\end{shdef}


\begin{shdef}
    \begin{definition}[Функция от матрицы через ряд Тейлора]
    \leavevmode \nl 

    Пусть \( f(t) \) — аналитическая функция в точке \( 0 \). Тогда эту функцию можно \\разложить в ряд Тейлора:
    \[
    f(t) = \sum_{k=0}^{\infty} \frac{f^{(k)}(0)}{k!} \, t^k,
    \]
    который сходится на некотором интервале сходимости \( (-R, R) \).

    Пусть \( \lambda_1, \ldots, \lambda_s \) — спектр матрицы оператора \( \phi \), и пусть все \( \lambda_j \) принадлежат \\интервалу сходимости \( (-R, R) \). Тогда функция от матрицы \( f(A) \) определяется как сумма ряда:
    \[
    f(A) = \sum_{k=0}^{\infty} \frac{f^{(k)}(0)}{k!} \, A^k.
    \]

    \textbf{Сходимость на спектре} означает, что для всех собственных значений \( \lambda_j \) \\матрицы \( A \) и для всех \( k = 0, 1, \ldots, h_j - 1 \) (где \( h_j \) — размерность наибольшей жордановой клетки для \( \lambda_j \)) выполняется:
    \[
    \left( \sum_{m=0}^{N} \frac{f^{(m)}(0)}{m!} \, A^m \right)^{(k)}(\lambda_j) \to f^{(k)}(\lambda_j) \quad \text{при } N \to \infty.
    \]
    \end{definition}
\end{shdef}


\begin{shdef}
    \begin{definition}[Норма матрицы и сходимость по норме]
    \leavevmode \nl 

    Пусть \( A_{n \times n} = (a_{ij}) \). Её норма (норма Фробениуса) определяется как:
    \[
    \|A\|_{2} = \sqrt{\sum_{i,j = 1}^{n} a_{ij}^2}.
    \]

    Последовательность матриц \( \{A_n\} \) \textbf{сходится по норме} к матрице \( A \), если
    \[
    \forall \varepsilon > 0 \quad \exists N(\varepsilon) \in \mathbb{N} \quad \forall n \geq N(\varepsilon) : \|A - A_n\|_{2} < \varepsilon.
    \]
    \end{definition}
\end{shdef}

\begin{shdef}
    \begin{definition}[Сходимость последовательности функций от матрицы по норме]
    \leavevmode \nl 
    
    Пусть \( f_n(t) \) и \( f(t) \) определены на спектре матрицы \( A \). 
    
    Говорят, что последовательность \( f_n(A) \) \textbf{сходится к \( f(A) \) по норме} (равномерно), если
    \[
    \|f_n(A) - f(A)\| \to 0 \quad \text{при } n \to \infty.
    \]
    \end{definition}
\end{shdef}