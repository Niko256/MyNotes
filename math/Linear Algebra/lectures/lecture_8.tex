\begin{proof}
    \leavevmode \nl 
    
    Пусть \( \lambda_{1}, \ldots, \lambda_{s} \) — все собственные числа оператора \( \phi \), и по условию они \\вещественные. Рассмотрим разложение пространства \( \mathbb{V} \) в прямую сумму корневых подпространств, соответствующих этим собственным значениям:
    \[
    \mathbb{V} = \mathcal{L}_{\lambda_{1}} \oplus \ldots \oplus \mathcal{L}_{\lambda_{s}},
    \]
    где \( \mathcal{L}_{\lambda_{i}} \) — корневое подпространство, соответствующее собственному значению \( \lambda_{i} \).

    По теореме 30 (о разложении корневого подпространства), каждое корневое \\подпространство \( \mathcal{L}_{\lambda_{i}} \) раскладывается в прямую сумму циклических подпространств:
    \[
    \mathcal{L}_{\lambda_{i}} = \mathbb{V}_{i1} \oplus \ldots \oplus \mathbb{V}_{ik_{i}},
    \]
    где \( \mathbb{V}_{ij} \) — циклическое подпространство, инвариантное относительно оператора \( \phi \).

    В каждом циклическом подпространстве \( \mathbb{V}_{ij} \) построим циклический базис \( \mathcal{E}^{ij} \), \\следуя теореме о базисе циклического подпространства. Пусть \( \mathbb{V}_{ij} \) порождается \\корневым вектором \( v_{ij} \) высоты \( m_{ij} \). Тогда базис \( \mathcal{E}^{ij} \) имеет вид:
    \[
    \mathcal{E}^{ij} = \left\{ v_{ij}, (\phi - \lambda_{i} I) v_{ij}, (\phi - \lambda_{i} I)^2 v_{ij}, \ldots, (\phi - \lambda_{i} I)^{m_{ij}-1} v_{ij} \right\}.
    \]
    Рассмотрим действие оператора \( \phi \) на элементы циклического базиса \( \mathcal{E}^{ij}:\)
    \[
    \phi(v_{ij}) = \lambda_{i} v_{ij} + (\phi - \lambda_{i} I) v_{ij},
    \]
    \[
    \phi\left((\phi - \lambda_{i} I) v_{ij}\right) = \lambda_{i} (\phi - \lambda_{i} I) v_{ij} + (\phi - \lambda_{i} I)^2 v_{ij},
    \]
    \[
    \vdots
    \]
    \[
    \phi\left((\phi - \lambda_{i} I)^{m_{ij}-1} v_{ij}\right) = \lambda_{i} (\phi - \lambda_{i} I)^{m_{ij}-1} v_{ij}.
    \]
    \\
    Таким образом, матрица оператора \( \phi \) в базисе \( \mathcal{E}^{ij} \) имеет вид жордановой клетки:
\[
J_{ij} = \begin{pmatrix}
    \lambda_{i} & 1 & 0 & \cdots & 0 \\
    0 & \lambda_{i} & 1 & \cdots & 0 \\
    \vdots & \vdots & \ddots & \ddots & \vdots \\
    0 & 0 & \cdots & \lambda_{i} & 1 \\
    0 & 0 & \cdots & 0 & \lambda_{i}
\end{pmatrix}.
\]
    Единички над диагональю в матрице \( J_{ij} \) возникают из-за того, как оператор \( \phi \) действует на базисные векторы \( \mathcal{E}^{ij} \). Рассмотрим подробнее:
    
    Оператор \( \phi \) переводит каждый базисный вектор в линейную комбинацию текущего и следующего вектора базиса. Например:
    \[
    \phi(v_{ij}) = \lambda_{i} v_{ij} + (\phi - \lambda_{i} I) v_{ij}.
    \]
    Здесь \( \lambda_{i} v_{ij} \) соответствует диагональному элементу \( \lambda_{i} \), а \( (\phi - \lambda_{i} I) v_{ij} \) — следующему базисному вектору, что дает единичку над диагональю.

\\

    Объединяя базисы \( \mathcal{E}^{ij} \) для всех циклических подпространств, получаем базис всего пространства \( \mathbb{V} \):
    \[
    \mathcal{E} = \bigcup_{i=1}^{s} \bigcup_{j=1}^{k_{i}} \mathcal{E}^{ij}.
    \]
    Поскольку подпространства \( \mathbb{V}_{ij} \) инвариантны и образуют прямую сумму, действие оператора \( \phi \) на векторы из одного подпространства \( \mathbb{V}_{ij} \) не затрагивает векторы из других подпространств. 

Таким образом, матрица оператора \( \phi \) в базисе \( \mathcal{E} \) будет иметь блочно-диагональную структуру, где каждый блок соответствует действию \( \phi \) на одно из циклических подпространств \( \mathbb{V}_{ij} \). Каждый блок — это жорданова клетка \( J_{ij} \), соответствующая подпространству \( \mathbb{V}_{ij} \).

Итак, матрица \( A_{\mathcal{E}}^{\phi} \) имеет вид:
    \[
    A_{\mathcal{E}}^{\phi} = \begin{pmatrix}
        J_{11} & 0 & \cdots & 0 \\
        0 & J_{12} & \cdots & 0 \\
        \vdots & \vdots & \ddots & \vdots \\
        0 & 0 & \cdots & J_{sk_{s}}
    \end{pmatrix},
    \]
    где каждая \( J_{ij} \) — жорданова клетка, соответствующая циклическому подпространству \( \mathbb{V}_{ij} \). Это и означает, что матрица \( A_{\mathcal{E}}^{\phi} \) имеет Жорданову форму.
\end{proof}


\begin{shcor}
    \begin{corollary}
        \leavevmode \nl 
        
        Размер жордановой клетки никогда не превышает алгебраическую кратность \\данного собственного числа $\lambda_{i}$
    \end{corollary}
\end{shcor}

\begin{shth}
    \begin{theorem}[О единственности Жордановой формы]
    \leavevmode \nl 
    
    Пусть \( \phi \in \mathcal{L}(\mathbb{V}, \mathbb{V}) \) — линейный оператор, и все собственные значения \( \lambda_i \) \\оператора \( \phi \) вещественные. Тогда Жорданова форма матрицы \( A_{\mathcal{E}}^{\phi} \) определена однозначно с точностью до порядка расположения Жордановых клеток. Более того, количество Жордановых клеток одного и того же размера, соответствующих каждому собственному числу, является инвариантом оператора \( \phi \).
    \end{theorem}
\end{shth}


\begin{proof}
    Рассмотрим собственное число \( \lambda \) оператора \( \phi \) и его Жорданову форму \( J \). Пусть:
    \begin{align*}
    S_{1} &= \text{количество клеток размера } 1 \times 1, \\
    S_{2} &= \text{количество клеток размера } 2 \times 2, \\
    &\vdots \\
    S_{p} &= \text{количество клеток размера } p \times p.
\end{align*}
    Нам нужно доказать, что числа \( S_{1}, S_{2}, \ldots, S_{p} \) не зависят от выбора Жорданова базиса.

    Введем обозначения для рангов степеней оператора \( \phi - \lambda I \):
    \[
    r_{1} = \text{rang}(\phi - \lambda I) = \text{rang}(J - \lambda E),
    \]
    \[
    r_{2} = \text{rang}(\phi - \lambda I)^2 = \text{rang}(J - \lambda E)^2,
    \]
    \[
    \vdots
    \]
    \[
    r_{p} = \text{rang}(\phi - \lambda I)^p = \text{rang}(J - \lambda E)^p.
    \]

    Заметим, что ранг оператора \( \phi \) равен количеству линейно независимых столбцов в матрице \( A_{\mathcal{E}}^{\phi} \). Это можно записать как:
    \[
    \text{rang}(\phi) = \dim(\text{Im}(\phi)).
    \]
    
    Рассмотрим Жорданову клетку размера \( k \times k \), соответствующую \( \lambda \):
    \[
    J_k(\lambda) = \begin{pmatrix}
    \lambda & 1 & 0 & \cdots & 0 \\
    0 & \lambda & 1 & \cdots & 0 \\
    \vdots & \vdots & \ddots & \ddots & \vdots \\
    0 & 0 & \cdots & \lambda & 1 \\
    0 & 0 & \cdots & 0 & \lambda
    \end{pmatrix}
    \ \overset{\text{Вычитаем } \lambda}{\leadsto} \ 
    J_k(\lambda) - \lambda E = \begin{pmatrix}
    0 & 1 & 0 & \cdots & 0 \\
    0 & 0 & 1 & \cdots & 0 \\
    \vdots & \vdots & \ddots & \ddots & \vdots \\
    0 & 0 & \cdots & 0 & 1 \\
    0 & 0 & \cdots & 0 & 0
    \end{pmatrix}
    \]
    Ранг этой матрицы равен \( k - 1 \), так как первые \( k - 1 \) строк линейно независимы, а последняя строка нулевая. Таким образом, для каждой клетки размера \( k \times k \):
    \[
    \text{rang}(J_k(\lambda) - \lambda E) = k - 1.
    \]
    
    Отсюда следует, что размер клетки \( k \) выражается через ранг как:
    \[
    k = \text{rang}(J_k(\lambda) - \lambda E) + 1.
    \]
    
    
    Если же вычесть \( \lambda \) из Жордановой клетки, соответствующей другому собственному числу, то матрица останется невырожденной, и её ранг не изменится.

    Общий размер всех клеток равен:
    
    \[
    n = \sum_{k=1}^p k \cdot S_k.
    \]
    Подставляя выражение для \( k \), получаем:
    \[
    n = \sum_{k=1}^p (\text{rang}(J_k(\lambda) - \lambda E) + 1) \cdot S_k.
    \]
    Раскрывая сумму, имеем:
    \[
    n = \sum_{k=1}^p \text{rang}(J_k(\lambda) - \lambda E) \cdot S_k + \sum_{k=1}^p S_k.
    \]
    Первая сумма равна \( r_1 \), так как \( r_1 \) — это общий ранг оператора \( \phi - \lambda I \). Вторая сумма — это общее количество клеток, соответствующих \( \lambda \). Таким образом:
    \[
    n = r_1 + S_1 + S_2 + \ldots + S_p.
    \]

    
    Теперь рассмотрим ранг \( r_2 = \text{rang}(J - \lambda E)^2 \). Для Жордановой клетки размера \( k \times k \):
    \[
    (J_k(\lambda) - \lambda E)^2 = \begin{pmatrix}
    0 & 0 & 1 & \cdots & 0 \\
    0 & 0 & 0 & \cdots & 0 \\
    \vdots & \vdots & \ddots & \ddots & \vdots \\
    0 & 0 & \cdots & 0 & 0 \\
    0 & 0 & \cdots & 0 & 0
    \end{pmatrix}.
    \]
    Рассмотрим разность рангов \( r_1 - r_2 \). Эта разность показывает, насколько \\уменьшился ранг при возведении оператора \( \phi - \lambda I \) в квадрат. Эта разность связана с количеством Жордановых клеток. 
    
    Для каждой Жордановой клетки размера \( k \times k \):
    \[
    \text{rang}(J_k(\lambda) - \lambda E) - \text{rang}(J_k(\lambda) - \lambda E)^2 = 
    \begin{cases}
        r_{1} - r_{2} = (k - 1) - (k - 2) = 1, & \text{если } k \geq 2, \\
        0, & \text{если } k = 1.
    \end{cases}
    \]
    Следовательно, общая разность \( r_1 - r_2 \) равна количеству клеток размера \( k \times k \) \(\; (k \geq 2)\):
    \[
    r_1 - r_2 = S_2 + S_3 + \ldots + S_p.
    \]

    
    Аналогично, для разностей \( r_{m-1} - r_m \) получаем:
    \[
    r_{m-1} - r_m = S_m + S_{m+1} + \ldots + S_p.
    \]
    Таким образом, система уравнений принимает вид:
    \[
    \begin{cases}
        n - r_1 = S_1 + S_2 + \ldots + S_p, \\
        r_1 - r_2 = S_2 + S_3 + \ldots + S_p, \\
        r_2 - r_3 = S_3 + S_4 + \ldots + S_p, \\
        \vdots \\
        r_{p-1} - r_p = S_p.
    \end{cases}
    \]

    Решим систему последовательно. Из последнего уравнения:
    \[
    S_p = r_{p-1} - r_p.
    \]
    Подставляя \( S_p \) в предпоследнее уравнение:
    \[
    r_{p-2} - r_{p-1} = S_{p-1} + S_p = S_{p-1} + (r_{p-1} - r_p),
    \]
    откуда:
    \[
    S_{p-1} = r_{p-2} - 2r_{p-1} + r_p.
    \]
    Продолжая аналогично, получаем общую формулу:
    \[
    S_k = r_{k-1} - 2r_k + r_{k+1} \quad \text{для } k = 1, 2, \ldots, p-1,
    \]
    где \( r_0 = n \) и \( r_{p+1} = 0 \).

    
    Поскольку ранги \( r_k \) не зависят от выбора базиса, числа \( S_k \), выраженные через \( r_k \), также не зависят от выбора базиса. Это доказывает, что количество Жордановых клеток каждого размера является инвариантом оператора \( \phi \).
\end{proof}

\begin{shcor}
    \begin{corollary}
        Пусть $\mathcal{L}^{\lambda}$ — собственное подпространство линейного оператора $\phi$, соответствующее собственному значению $\lambda$. Тогда:
        \[
        \dim(\mathcal{L}^{\lambda}) = s(\lambda) = j(\lambda),
        \]
        где:
        \begin{itemize}
            \item $s(\lambda)$ — количество линейно независимых собственных векторов, соответствующих $\lambda$;
            \item $j(\lambda)$ — количество Жордановых клеток, соответствующих $\lambda$.
        \end{itemize}
    \end{corollary}
\end{shcor}

\begin{proof}
    Рассмотрим оператор $\phi - \lambda I$, где $I$ — тождественный оператор. Из классического курса линейной алгебры известно:
    \[
    \dim(\ker(\phi - \lambda I)) + \dim(Im(\phi - \lambda I)) = n,
    \]
    где $n$ — размерность пространства.

    Заметим, что:
    \begin{itemize}
        \item $\ker(\phi - \lambda I)$ — это собственное подпространство $\mathcal{L}^{\lambda}$, соответствующее $\lambda$.
        \item $\dim(\ker(\phi - \lambda I)) = \dim(\mathcal{L}^{\lambda})$.
        \item $\dim(Im(\phi - \lambda I)) = r_1$, где $r_1$ — ранг оператора $\phi - \lambda I$.
    \end{itemize}

    Таким образом, получаем:
    \[
    \dim(\mathcal{L}^{\lambda}) + r_1 = n.
    \]

    Пусть \( S_1, S_2, \ldots, S_p \) — количество Жордановых клеток размера \( 1 \times 1, 2 \times 2, \ldots, p \times p \) соответственно, соответствующих собственному значению \( \lambda \). Тогда общее количество Жордановых клеток, соответствующих \( \lambda \), равно:
    \[
    j(\lambda) = S_1 + S_2 + \ldots + S_p.
    \]

    Каждая Жорданова клетка размера \( k \times k \) вносит ровно один линейно независимый собственный вектор. Поэтому количество линейно независимых собственных векторов, соответствующих \( \lambda \), равно:
    \[
    s(\lambda) = S_1 + S_2 + \ldots + S_p = j(\lambda).
    \]

    С другой стороны, размерность собственного подпространства \( \mathcal{L}^{\lambda} \) также равна количеству линейно независимых собственных векторов, то есть:
    \[
    \dim(\mathcal{L}^{\lambda}) = s(\lambda) = j(\lambda).
    \]
\end{proof}

\begin{shth}
    \begin{theorem}
        \leavevmode \nl 
        
Пусть $\phi \in \mathcal{L}(\mathbb{V}, \mathbb{V})$ — линейный оператор, все корни характеристического \\уравнения которого вещественные. 

    Если $\lambda_{0}$ — собственное число алгебраической кратности $l_{0}$, то 
    \[
        \dim(\mathcal{L}_{\lambda_{0}}) = l_{0},
    \]
    где $\mathcal{L}_{\lambda_{0}}$ — корневое подпространство, соответствующее $\lambda_{0}$.
    \end{theorem}
\end{shth}



\begin{proof}
Рассмотрим корневое подпространство $\mathcal{L}_{\lambda_{0}}$, соответствующее \\собственному числу $\lambda_{0}$.

Пусть $J$ — жорданова форма оператора $\phi$, состоящая из жордановых клеток, \\соответствующих собственным числам. Для собственного числа $\lambda_{0}$ обозначим:

\begin{itemize}
    \item $K_{1}$ — количество жордановых клеток размера $1 \times 1$ с $\lambda_{0}$ на диагонали,
    \item $K_{2}$ — количество жордановых клеток размера $2 \times 2$ с $\lambda_{0}$ на диагонали,
    \item $\vdots$
    \item $K_{s}$ — количество жордановых клеток размера $s \times s$ с $\lambda_{0}$ на диагонали.
\end{itemize}

Тогда жорданова форма $J$ может быть записана как блочно-диагональная матрица:
\[
J = \begin{pmatrix}
J_{1} & 0 & \ldots & 0 \\
0 & J_{2} & \ldots & 0 \\
\vdots & \vdots & \ddots & \vdots \\
0 & 0 & \ldots & J_{s}
\end{pmatrix}
\]
где каждый блок $J_{i}$ — жорданова клетка, соответствующая $\lambda_{0}$.

Корневое подпространство $\mathcal{L}_{\lambda_{0}}$ раскладывается в прямую сумму циклических \\подпространств, каждое из которых соответствует одной жордановой клетке. 

Следовательно, размерность $\mathcal{L}_{\lambda_{0}}$ равна сумме размеров всех жордановых клеток, соответствующих $\lambda_{0}$:
\[
\dim(\mathcal{L}_{\lambda_{0}}) = K_{1} + 2K_{2} + \ldots + sK_{s}.
\]

Характеристический многочлен оператора $\phi$ имеет вид:
\[
\chi_{\phi}(t) = \det(J - tE) = (\lambda_{0} - t)^{K_{1}} (\lambda_{0} - t)^{2 K_{2}} \cdot \ldots \cdot (\lambda_{0} - t)^{s K_{s}} = (\lambda_{0} - t)^{K_{1} + 2K_{2} + \ldots + sK_{s}}
\]
Алгебраическая кратность $l_{0}$ собственного числа $\lambda_{0}$ равна сумме размеров всех жордановых клеток, соответствующих $\lambda_{0}$:
\[
l_{0} = K_{1} + 2K_{2} + \ldots + sK_{s}
\]

Таким образом, мы получаем:
\[
\dim(\mathcal{L}_{\lambda_{0}}) = l_{0}
\]
что и требовалось доказать.
\end{proof}

\vspace{0.2cm}

\begin{shth}
    \begin{theorem}[О минимальном многочлене для Жордановой матрицы]
        \leavevmode \nl 
        
        Пусть \( J_{\phi} \) — жорданова форма матрицы оператора \(\phi\), представленная в виде блочно-диагональной матрицы.
        
    Тогда минимальный многочлен \( m_{\phi}(t) \) оператора \(\phi\) (или матрицы \( J_{\phi} \)) равен \\наименьшему общему кратному (НОК) минимальных многочленов жордановых клеток \( J_{1}, J_{2}, \ldots, J_{r} \):
    \[
    m_{\phi}(t) = m_{J_{\phi}}(t) = \text{НОК}\{m_{J_{1}}(t), m_{J_{2}}(t), \ldots, m_{J_{r}}(t)\}.
    \]

    Если \( J_{i} \) — жорданова клетка размера \( m \times m \), соответствующая собственному числу \(\lambda\), то её минимальный многочлен имеет вид:
    \[
    m_{J_{i}}(t) = (t - \lambda)^m.
    \]

    Кроме того, корневые высоты векторов, соответствующих этой жордановой клетке, равны \( 1, 2, \ldots, m \).
    \end{theorem}
\end{shth}

\begin{proof}
Докажем, что минимальный многочлен \( m_{k}(t) \) жордановой \\клетки \( K \) размера \( m \times m \), соответствующей собственному числу \( \lambda \), равен \( (t - \lambda)^m \).

Рассмотрим жорданову клетку \( K \) размера \( m \times m \):
\[
K = \begin{pmatrix}
\lambda & 1 & 0 & \ldots & 0 \\
0 & \lambda & 1 & \ldots & 0 \\
\vdots & \vdots & \ddots & \ddots & \vdots \\
0 & 0 & \ldots & \lambda & 1 \\
0 & 0 & \ldots & 0 & \lambda
\end{pmatrix}
\]
Покажем, что многочлен \( (t - \lambda)^m \) аннулирует клетку \( K \), то есть:
\[
(K - \lambda I)^m = 0
\]

Для этого рассмотрим действие \( K - \lambda I \) на стандартные базисные векторы \( e_1, e_2, \dots, e_m \), где \( e_i \) — вектор с единицей на \( i \)-й позиции и нулями на остальных.


Матрица \( K - \lambda I \) имеет вид:
\[
K - \lambda I = \begin{pmatrix}
0 & 1 & 0 & \ldots & 0 \\
0 & 0 & 1 & \ldots & 0 \\
\vdots & \vdots & \ddots & \ddots & \vdots \\
0 & 0 & \ldots & 0 & 1 \\
0 & 0 & \ldots & 0 & 0
\end{pmatrix}
\]

Действие \( K - \lambda I \) на базисные векторы:
\[
(K - \lambda I) e_1 = 0, \quad (K - \lambda I) e_2 = e_1, \quad (K - \lambda I) e_3 = e_2, \quad \ldots, \quad (K - \lambda I) e_m = e_{m-1}.
\]
Таким образом, \( K - \lambda I \) "сдвигает" каждый базисный вектор на одну позицию вверх, за исключением \( e_1 \), который обнуляется.

Эта матрица является нильпотентной степени \( m \), то есть \( (K - \lambda I)^m = 0 \), но \\\( (K - \lambda I)^{m-1} \neq 0 \). Следовательно, \( (t - \lambda)^m \) — аннулирующий многочлен для \( K \).

Покажем, что \( (t - \lambda)^m \) является минимальным многочленом для \( K \). Предположим, что существует многочлен \( p(t) = (t - \lambda)^s \), где \( s < m \), который также аннулирует \( K \). Тогда:
\[
(K - \lambda I)^s = 0.
\]
Однако, как было показано выше, \( (K - \lambda I)^{m-1} \neq 0 \), и, следовательно, \( s \) не может быть меньше \( m \). Таким образом, минимальный многочлен \( m_{k}(t) \) равен \( (t - \lambda)^m \).


Матрица \( K - \lambda I \) действует на стандартные базисные векторы \( e_1, e_2, \ldots, e_m \) следующим образом:
\[
(K - \lambda I) e_1 = 0, \quad (K - \lambda I) e_2 = e_1, \quad \ldots, \quad (K - \lambda I) e_m = e_{m-1}.
\]
Отсюда видно, что:
\begin{itemize}
    \item Вектор \( e_1 \) имеет корневую высоту \( 1 \), так как \( (K - \lambda I) e_1 = 0 \).
    \item Вектор \( e_2 \) имеет корневую высоту \( 2 \), так как \( (K - \lambda I)^2 e_2 = 0 \), но \( (K - \lambda I) e_2 \neq 0 \).
    \item \(\vdots\)
    \item Вектор \( e_m \) имеет корневую высоту \( m \), так как \( (K - \lambda I)^m e_m = 0 \), но \\\( (K - \lambda I)^{m-1} e_m \neq 0 \).
\end{itemize}

Таким образом, минимальный многочлен \( m_K(t) \) равен \( (t - \lambda)^m \), так как:
\begin{itemize}
    \item \( (t - \lambda)^m \) аннулирует \( K \), то есть \( (K - \lambda I)^m = 0 \).
    \item Никакой многочлен меньшей степени \( s < m \) не может аннулировать \( K \), так как \( (K - \lambda I)^{m-1} e_m \neq 0 \).
\end{itemize}

Теорема доказана.
\end{proof}

\begin{shth}
    \begin{theorem}
    Пусть \( J \) — Жорданова форма матрицы оператора \(\phi\), и \(\lambda_1, \lambda_2, \ldots, \lambda_s\) — её собственные числа. Обозначим через \( h_j \) максимальный размер Жордановой клетки, соответствующей собственному числу \(\lambda_j\). Тогда минимальный многочлен \( m_{\phi}(t) \) оператора \(\phi\) имеет вид:
    \[
    m_{\phi}(t) = (t - \lambda_1)^{h_1} \cdot (t - \lambda_2)^{h_2} \cdot \ldots \cdot (t - \lambda_s)^{h_s}.
    \]
    \end{theorem}
\end{shth}

\begin{proof}
    Минимальный многочлен \( m_{\phi}(t) \) оператора \(\phi\) совпадает с минимальным многочленом его Жордановой формы \( J \). Жорданова форма \( J \) состоит из Жордановых клеток \( J_1, J_2, \ldots, J_r \), каждая из которых соответствует одному из собственных чисел \(\lambda_1, \lambda_2, \ldots, \lambda_s\).

    Для каждой Жордановой клетки \( J_i \), соответствующей собственному числу \(\lambda_j\), \\минимальный многочлен равен \( m_{J_i}(t) = (t - \lambda_j)^{k_i} \), где \( k_i \) — размер клетки \( J_i \). 
    
    Следовательно, минимальный многочлен всей Жордановой формы \( J \) равен \\наименьшему общему кратному (НОК) минимальных многочленов всех Жордановых клеток:
    \[
    m_{\phi}(t) = \text{НОК}\{m_{J_1}(t), m_{J_2}(t), \ldots, m_{J_r}(t)\}.
    \]

    Поскольку \( h_j \) — это максимальный размер Жордановой клетки, соответствующей собственному числу \(\lambda_j\), то для каждого \(\lambda_j\) минимальный многочлен \( m_{\phi}(t) \) должен содержать множитель \( (t - \lambda_j)^{h_j} \). Таким образом:
    \[
    m_{\phi}(t) = \text{НОК}\{(t - \lambda_1)^{h_1}, (t - \lambda_2)^{h_2}, \ldots, (t - \lambda_s)^{h_s}\}.
    \]

    Так как собственные числа \(\lambda_1, \lambda_2, \ldots, \lambda_s\) различны, многочлены \((t - \lambda_{j})^{h_{j}\) \\взаимно просты, а значит, что НОК многочленов \( (t - \lambda_1)^{h_1}, (t - \lambda_2)^{h_2}, \ldots, (t - \lambda_s)^{h_s} \) равно их произведению:
    \[
    m_{\phi}(t) = (t - \lambda_1)^{h_1} \cdot (t - \lambda_2)^{h_2} \cdot \ldots \cdot (t - \lambda_s)^{h_s}.
    \]
\end{proof}

\vspace{0.3cm}

\begin{shcor}
    \begin{corollary}
    \leavevmode \nl 
    
    Пусть $\phi \in \mathcal{L}(\mathbb{V}, \mathbb{V})$ и все корни характеристического уравнения вещественные. Тогда существует базис из собственных векторов $\phi$ тогда и только тогда, когда все корни минимального многочлена не кратные.
    \end{corollary}
\end{shcor}

\begin{proof}
\leavevmode \nl 

    \boxed{\Rightarrow} Предположим, что существует базис из собственных векторов оператора $\phi$. Это означает, что каждое корневое подпространство имеет высоту 1, то есть все жордановы клетки имеют размер $1 \times 1$. Минимальный многочлен $m_{\phi}(t)$ в этом случае имеет вид:
    \[
    m_{\phi}(t) = (t - \lambda_1) \cdot \ldots \cdot (t - \lambda_n),
    \]
    где $\lambda_1, \ldots, \lambda_n$ — собственные значения оператора $\phi$. Таким образом, все корни минимального многочлена не кратные.

    \boxed{\Leftarrow} Обратно, предположим, что все корни минимального многочлена $m_{\phi}(t)$ не кратные. Тогда минимальный многочлен имеет вид:
    \[
    m_{\phi}(t) = (t - \lambda_1) \cdot \ldots \cdot (t - \lambda_n),
    \]
    где $\lambda_1, \ldots, \lambda_n$ — различные собственные значения. Это означает, что все жордановы клетки имеют размер $1 \times 1$, и, следовательно, все корневые подпространства имеют высоту 1. Таким образом, существует базис из собственных векторов оператора $\phi$.
\end{proof}